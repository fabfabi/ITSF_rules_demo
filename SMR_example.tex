%---------------------------------------------------------------------------
\documentclass[fontsize=11pt, paper=a4]{report}
%---------------------------------------------------------------------------

\usepackage[english]{babel}  
\usepackage{url}
\usepackage{lipsum}
\usepackage[utf8]{inputenc}

%---------------------------------------------------------------------------
%\usepackage{arev}
\usepackage[T1]{fontenc}
\usepackage[notmath]{sansmathfonts}

%\renewcommand{\sfdefault}{fav}

%\usepackage{helvet} 
%\renewcommand{\familydefault}{\sfdefault} %delete serifs

%---------------------------------------------------------------------------
%\usepackage{clipboard} %copy the definitions via \copy{key}{text} and \paste
%---------------------------------------------------------------------------
\usepackage{hyperref}%needs to be loaded BEFORE glossaries
\usepackage[automake]{glossaries}
%\usepackage[toc, section=section]{glossaries} %makes the entry chapter below a section

%\usepackage{microtype}
%\DisableLigatures{encoding=*,family=*}

%---------------------------------------------------------------------------
\usepackage{amsthm} % for Definitions and Theoremes
%\theoremstyle{plain}

%---------------------------------------------------------------------------
\usepackage{titlesec} %to not say `Chapter i' in front of every chapter
\titleformat{\chapter}{\normalfont\huge}{\thechapter.}{20pt}{\huge\bf}

%---------------------------------------------------------------------------
% Formatting for the definitions and their references
\newcommand{\refformat}[1]{{\sffamily \textsc{#1}}}			 %formatting of references to definitions
\newcommand{\defformat}[1]{{\textbf{\refformat{#1}}}}%formatting of definitions --> References in bold face

%---------------------------------------------------------------------------
\usepackage{thmtools} % for Definitions and Theoremes
\declaretheoremstyle[
    spaceabove=6pt, 
    spacebelow=6pt, 
    headfont=\normalfont\bfseries,
    notefont=\mdseries\bfseries, 
    notebraces={}{},%to not have anything around the name
    bodyfont=\normalfont,
    postheadspace=1em,
    headpunct={:}]{defstyle}
\declaretheorem[numbered=no, style = defstyle]{Definition}
\renewcommand{\listtheoremname}{List of Definitions} %needed in order to have it in the title for \listoftheorems

% The formatting of the Definintions environment
\newcommand{\dtterm}[1]{
%{\defformat{#1}: #2}
\begin{Definition}[name=\defformat{#1}]
	\label{d:#1}
	 \glsdesc*{#1} %Star for ``no hyperlink''
\end{Definition}
}



%\def\UrlFont{\normalfont}%formatting links to normal - this might be needed in case of hyperref

%---------------------------------------------------------------------------
% Actual Code for the Defining the terms for the glossary

\newcommand{\gtterm}[2]{%Initial definition for the glossary
\label{#1}
	%\glstarget{##1}{%
		\newglossaryentry{#1}{%
				name=#1,%
				description = {#2}%
			}%
		%}%
}


%---------------------------------------------------------------------------
%References within the document within the documents
%\newcommand{\dref}[1]{\refformat{\nameref{d:#1}}} %ref to definition

%to Definition:
\newcommand{\dref}[1]{\refformat{\hyperref[d:#1]{#1}}} 
%To technical Term
\newcommand{\tref}[1]{\refformat{{#1}}} %Ref to technical term (general)
%to glossary
\newcommand{\gref}[1]{\refformat{\gls{#1}}} %Ref to Glossary




%---------------------------------------------------------------------------
%---------------------------------------------------------------------------
%Glossaries stuff
\renewcommand{\glsnamefont}[1]{\defformat{\hyperref[d:#1]{#1}}} %Defines the name in the glossary

\newcommand{\tterm}[1]{% this one is used for the 
	\defformat{#1}: \glsdesc*{#1} %star removes hyperlink
}
\newcommand{\recap}[1]{\item{\tterm{#1}}}
%--------------------------------------------------------------------------

%---------------------------------------------------------------------------
% Define all the technical Terms
\makeglossaries
\gtterm{stationary ball}{A ball which is not moving}
\gtterm{contact}{When the ball touches a figure on a rod, or any part of a rod that is internal to the playing field}
\gtterm{move}{A \tref{contact} that causes a \tref{stationary ball} to move, or a moving ball to change speed or direction}
\gtterm{team}{One or more players on the same side of the table}

%---------------------------------------------------------------------------
%---------------------------------------------------------------------------
%---------------------------------------------------------------------------
%---------------------------------------------------------------------------
%Start the Document
\begin{document}
\sffamily %without serifs


%---------------------------------------------------------------------------
\title{SMR technical demo for \LaTeX}
\maketitle

\tableofcontents

%---------------------------------------------------------------------------
\pagestyle{plain}%% keine Header in der Kopfzeile bzw. plain
\pagenumbering{arabic}

%---------------------------------------------------------------------------
\chapter{First Chapter}
Just as an example, this can be the first chapter. This Chapter contains the original two big definitions for the following terms.
\dtterm{stationary ball}
\dtterm{contact}

\newpage

%---------------------------------------------------------------------------
\chapter{Second Chapter}
Useful definitions
\begin{enumerate}
	\recap{stationary ball}
	\recap{contact}
	\recap{team}
\end{enumerate}
Now there is a second chapter that may reference some of these definitions. It also defines a new one...
\dtterm{move}
 And later there is some plain text, that where \gref{move} links to the glossary and \dref{stationary ball} links to the original definiton.\\
 The Glossary Entries link to the original definitions. Except for \gref{team} that does not have such a definition.

\chapter{How it works}
There are several macros defined. For example there is a macro for a definition and a reference for a definition. If you change the formatting for the macro, it changes throughout the entire document.\\
All glossary entries are defined before the actual document starts (invisible). Lateron they are just copied to where they are needed. So if you want to have the term \tref{move} in the `useful definitions' part, the description will be automatically added via the macro recap(move). If the original definition is changed, all related instances will be changed as well.\\
The hyperlinks are generated automatically based on which macro you use. You can use the definition of \tref{move} without the link, you can reference \dref{move} with a link to the definition or you can reference \gref{move} with a link to the glossary.

%---------------------------------------------------------------------------
\clearpage
\printglossary[title=Definitions, toctitle=List of terms]%, nonumberlist]
%\printnoidxglossary
%---------------------------------------------------------------------------
\end{document}
%
