%---------------------------------------------------------------------------
\documentclass[fontsize=11pt, paper=a4]{report}
%---------------------------------------------------------------------------
\usepackage[T1]{fontenc}
\usepackage[english]{babel}  
\usepackage{url}
\usepackage{lipsum}
\usepackage[utf8]{inputenc}
%\usepackage{clipboard} %copy the definitions via \copy{key}{text} and \paste
\usepackage[sanitize=none]{glossaries}
\usepackage{amsthm} % for Definitions and Theoremes
\usepackage{thmtools} % for Definitions and Theoremes
%\declaretheorem[numbered=no]{Definition}
%---------------------------------------------------------------------------

%\newcommand{\tterm}[2]{\hypertarget{#1}{\defformat{#1}}:#2}
%\newcommand{\tref}[1]{\hyperlink{#1}{\refformat{#1}}}
%---------------------------------------------------------------------------

%\renewcommand{\listtheoremname}{List of Definitions} %needed in order to have it in the title for \listoftheorems

\usepackage{hyperref}
\usepackage{cleveref}

%---------------------------------------------------------------------------
% Formatting for the definitions and their references
\newcommand{\refformat}[1]{{\textsc{#1}}}			 %formatting of references to definitions
\newcommand{\defformat}[1]{{\textbf{\refformat{#1}}}}%formatting of definitions --> References in bold face

%---------------------------------------------------------------------------
% Actual Code for the Definitions and the references

%\newcommand{\tterm}[2]{{\defformat{#1}: #2}} % this one works :-)

\newcommand{\dtterm}[2]{%Initial definition for the glossary
%	{\defformat{#1}: #2}
	\newglossaryentry{#1}{
			name=#1,
			description = {#2}
		}
}
\newcommand{\tref}[1]{\refformat{{#1}}}

\newcommand{\tterm}[1]{
\defformat{#1}: \glsdesc{#1}
}


%---------------------------------------------------------------------------

%---------------------------------------------------------------------------
%---------------------------------------------------------------------------

\makeglossaries
\dtterm{stationary ball}{A ball which is not moving.}
\dtterm{contact}{When the ball touches a figure on a rod, or any part of a rod that is internal to the playing field.}
\dtterm{move}{A \tref{contact} that causes a \tref{stationary ball}}
\begin{document}


%---------------------------------------------------------------------------
\title{SMR technical demo for \LaTeX}
\maketitle
%\author{someone}

%\listoftheorems
%---------------------------------------------------------------------------
\pagestyle{plain}%% keine Header in der Kopfzeile bzw. plain
\pagenumbering{arabic}
%---------------------------------------------------------------------------
\chapter*{First Chapter}
Just as an example, this can be the first chapter. This Chapter contains the original two definitions for the following terms
\tterm{stationary ball}
\tterm{contact}

\newpage
%---------------------------------------------------------------------------
\chapter*{Second Chapter}
Important Definitions
\begin{enumerate}
	\item{\tterm{stationary ball}}
\end{enumerate}

Now there is a second chapter that may reference some of these definitions
\tterm{move}
Please note, that the references from the definition link to the original definition.

%---------------------------------------------------------------------------
\newpage
\chapter*{Third Chapter}
Important Definitions
\begin{enumerate}
	\item{\tterm{move}}
	\item{\tterm{stationary ball}}
\end{enumerate}
Some Plain text, that just references to \tref{move} and \tref{stationary ball}.

%---------------------------------------------------------------------------
\clearpage

\printglossary%[title=Definitions, toctitle=List of terms]
%---------------------------------------------------------------------------
\end{document}
%
